\chapter{Estado del arte}
En este capítulo realizará un análisis del entorno relacionado con el problema a solucionar y las soluciones ya existentes.

\section{Búsqueda y análisis de información}
Para facilitar la comprensión de la gran cantidad de conceptos y definiciones relacionadas con el trabajo se ha organizado la investigación de tal forma que se ha partido de un alto nivel de abstracción y aclaración de términos generales para así poder llegar a especificaciones técnicas de más bajo nivel.

Las fuentes de información han sido muy variadas, partiendo de artículos de investigación, pasando por documentación oficial y foros de comunidad de las tecnologías y herramientas usadas, así como el uso de recursos multimedia como pueden ser vídeos explicativos a niveles teóricos y prácticos. Además, se ha realizado una búsqueda de sistemas parecidos al propuesto para poder partir de una base.

Toda la información se ha estudiado desde un punto crítico, para así intentar que el desarrollo del trabajo sea lo más eficiente posible y así poder evitar el uso de malas prácticas.

\section{Conceptos}
Una vez estructurada toda la información recogida, se pueden definir los siguientes conceptos necesarios para entender el diseño y la implementación del sistema
\subsection{Computación Perimetral de Acceso Múltiple}
\subsubsection{Definición}
La Computación Perimetral de Acceso Múltiple o arquitectura de red Multi-Access Edge Computing (MEC) genera nuevas capacidades de computación en la nube al estar cerca de los usuarios. Se crea un entorno de poca latencia y mucho ancho de banda dando así la posibilidad de poder desarrollar aplicaciones más potentes, con servicios personalizados con contexto propio para cada uno de los usuarios.

Como podemos ver en la figura, se pueden diferenciar tres partes, los dispositivos, aquellas aplicaciones de los usuarios que están conectadas con los nodos del borde llamado Edge, estos están cerca de los usuarios en lo que la geolocalización se refiere. Estos pueden realizar procesamiento de datos a tiempo real y análisis de datos entre otras funcionalidades. Son estos nodos los que están conectados con la nube o centros de datos llamados Core.

Existen muchos casos donde se utilice la computación perimetral, algunos de ellos pueden ser en el sector industrial haciendo uso de análisis de datos en tiempo real para así evitar fallos de funcionamiento o reparaciones innecesarios. Otro ejemplo puede ser en el sector de seguridad y vigilancia, puesto a que los datos procesados en tiempo real mejorarían considerablemente.

Uno de los sectores donde es más notorio el crecimiento de esta arquitectura es en la industria del entretenimiento. La computación perimetral se usa para ofrecer a los clientes una experiencia e inmersión aún mayor, en áreas tales como deportes, eventos y actuaciones.

\subsubsection{Futuro de la computación perimetral}
Esta arquitectura ha ganado mucha fuerza durante los últimos años debido al impulso del uso del Internet de las cosas, las redes 4G y las redes 5G de última generación. Además, la gran evolución de la realidad aumentada gracias la inteligencia artificial junto a los vídeos en tiempo real y junto a la conexión entre dispositivos en masa favorece el estudio de dicha arquitectura y a su constante mejora.


\subsection{Algoritmo de detección de personas}

\section{Soluciones ya existentes}