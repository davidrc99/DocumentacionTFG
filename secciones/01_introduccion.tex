\chapter{Introducción}

\section{Motivación del trabajo}
Es un hecho que en la última década se han hecho grandes avances en el sector tecnológico y comunicaciones. Cosas tales como compartir el contenido en la nube o controlar dispositivos con el teléfono móvil se han convertido en parte de nuestra vida cotidiana, así como que en un futuro también lo podrán ser el uso de vehículos totalmente autónomos y casas inteligentes.

Sin embargo, este incremento en el uso de dispositivos interconectados y el aumento de volumen del tráfico provoca que se deba de mejorar el procesamiento de información en tiempo real para poder ofrecer al usuario final una respuesta en el menor tiempo posible, para ello una solución puede ser el uso de la computación perimetral y tecnologías 5G.

\subsection{Computación perimetral y redes 5G}
La computación perimetral o edge computing es un paradigma que acerca la computación y el almacenamiento de los datos al sitio donde se necesite para que así se pueda mejorar los tiempos de respuesta y poder ahorrar ancho de banda.

Si este paradigma se trabaja junto a tecnologías asociadas a la quinta generación de redes móviles conocida como redes 5G se puede obtener no solo una gran mejora en las conexiones de los dispositivos, sino que también se podrían implementar aplicaciones que requieran una gran cantidad de computación con tiempo de respuesta aceptables.

\subsection{Realidad Aumentada}
Aunque la primera implementación tecnológica basada en realidad aumentada haya sido en 1957, la sociedad no ha podido experimentar de primera mano las aplicaciones de realidad aumentada hasta hace unos años debido a que ha sido ahora cuando los dispositivos móviles ofrecen las prestaciones suficientes para ser compatibles con esta tecnología.

El uso de la realidad aumentada actualmente se puede ver en muchos ámbitos diferentes como en el sector sanitario, industrial o educativo, a la vez que se va desarrollando en otros sectores diferentes.

\section{Descripción del problema}
En esta era digital donde las tecnologías de información y comunicación ya forman parte de la vida normal de las personas ha provocado que haya una demanda cada vez más alta de aplicaciones que faciliten y hagan más amena la vida de la sociedad.

Estas aplicaciones tendrán la exigencia por parte de sus usuarios de ser capaces de realizar tareas más complejas de las actuales sin perder su eficiencia y en el mejor de los casos, mostrando los resultados de la forma más intuitiva, rápida y gráfica posible. Además, tendrán que ser capaces de poder recibir peticiones en masa de una gran cantidad de usuarios en un periodo muy pequeño de tiempo, por lo que la conexión entre el dispositivo final y el servidor debe de ser lo más depurado y eficiente posible.

\subsection{Escenario simulado}
Para poder afrontar un problema de esta índole, se ha simulado un escenario real que podemos ver en nuestra vida cotidiana, un partido de fútbol.

En un partido de fútbol de equipos de primera división, como mínimo asisten de media unos 30.000 espectadores. Dependiendo del asiento comprado, se puede tener una mejor o peor visión del partido, por ejemplo, si un espectador acude a un partido y se sienta detrás de una de las porterías, lo más seguro es que no pueda apreciar lo que sucede cuando el balón esté en el área de la otra portería. Para ello se deberá de buscar una solución en la que todos los espectadores que acudan al campo puedan tener una visión general del partido independientemente de donde estén sentados.

\subsection{Objetivos del trabajo}
En este proyecto se pretende diseñar, desarrollar y evaluar un servicio orientado al escenario simulado anteriormente mencionado, creando una aplicación que aproveche las ventajas que ofrecen la computación perimetral y tecnologías asociadas a redes 5G. 

\section{Estructura del documento}

La estructura de la memoria del trabajo está dividida en los siguientes capítulos: tras el Capítulo 1 de introducción, en el Capítulo 2 se realiza un estudio del estado del arte, analizando la información buscada por diferentes vías y aclarando por consecuencia una serie de conceptos para la comprensión del diseño y desarrollo del sistema.
En el Capítulo 3 se explica la metodología usada para realizar un seguimiento del trabajo, se hace un análisis de los requisitos que debe de tener el sistema y se realiza una expeculación de su coste final.
El diseño de diferentes soluciones al problema está especificado en el Capítulo 4. A continuación, en el Capítulo 5 se describe todo el proceso de implementación de dichas soluciones junto a las diferencias que hay entre ellas. 
Seguidamente en el Capítulo 6, se muestra una serie de pruebas usando las diferentes versiones del sistema y se analiza dichos resultados para poder realizar una comparación entre ellas. Finalmente se llegará a una conclusión del trabajo realizado en el Capítulo 7 junto a las líneas futuras de investigación que se abren tras este trabajo. En la último capítulo se expone las referencias bibliográficas consultadas durante todo el trabajo. 